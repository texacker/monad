%% Generate document with:
%% bibtex monad && [xe]latex -src-specials monad
\documentclass{article}

%% Charset & Font:
%\usepackage{ctex}          % CJK
%\usepackage{mathptmx}      % mathptmx font
%\usepackage{tgpagella}     % Palatino, text only
\usepackage{mathpazo}       % Palatino, math & text

\usepackage{amsmath,amsfonts,amssymb,amsbsy}
\usepackage{amsthm}

\usepackage{natbib}

\input{../../include/usrdef}

\begin{document}

%\title{\textbf{Norm Minimization of Linear Matrix}}
%\author{
%Rao Fu \\[1em]
%\small{\textit{Center for Systems and Control}} \\
%\small{\textit{Department of Mechanics and Engineering Science}} \\
%\small{\textit{Peking University, Beijing 100871, P.R.China}}
%}

%\title{\Huge{\textbf{LECTURE NOTES}} \\ \Huge{\textbf{ON}} \\ \Huge{\textbf{MATHEMATICS}} \\[1em] {Note on Monad}}

\title{\textbf{Lecture Notes on Mathmatics} \\[1em] Note on Monad
{\footnote{\texttt{Document No. 2017-09-22-001}}}
{\footnote{\copyright \texttt{2017 - 2017 by texacker.org. All rights reserved.}}}
}

\author{%
%\textit{texacker.org}
\small{\textit{rao.fu}}\thanks{\texttt{http://www.texacker.org/\~{}furao}} \\ \small{\textit{texacker.org}}
}
\date{\today}

\maketitle

\section{Introduction}
The problem to minimize the (spectral or maximum singular value) norm of $\norm{A}$
can recast as an LMI problem:
\begin{gather}
%\label{equ:SDP.primal}
\min \gamma \\
s.t.\,\,
\begin{bmatrix}
\gamma I & A \\
A^T      & \gamma I
\end{bmatrix}
\ge 0 \notag
\end{gather}
therefore the minimization of $\norm{P - P_0}$ with given $P_0$
has the form:
\begin{gather}
%\label{equ:SDP.primal}
\min \gamma \\
s.t.\,\,
\begin{bmatrix}
\gamma I    & P - P_0 \\
P^T - P_0^T & \gamma I
\end{bmatrix}
\ge 0 \notag
\end{gather}

My research interests mainly focus on the following areas:
\begin{itemize}
\item
Convex optimization approach for structured singular value analysis and synthesis;
\item
Object-oriented parallel SDP algorithm and software for large-scale robust control;
\item
LMI-based absolute stabilization of uncertain nonlinear system (Lur'e system \& Pendulum-like System).
\end{itemize}

\section{Previous Research}
\subsection{Convex optimization approach for large-scale SSV analysis and synthesis}
It's well known that the structured singular value (SSV) method,
which was introduced by {\citet{Doyle82A}} and {\citet{Safonov82}},
is one of most effective approach to deal with robustness analysis and controller synthesis for systems
subject to mixed un-modelled dynamic uncertainty and parametric uncertainty.
However, except for some special cases, most of the currently available approaches for SSV analysis and synthesis require
frequency sweeping which is computational costly and prone to erroneous answers
when the SSV curve is discontinuous.
Besides, the curve-fitting step involved in the synthesis procedure makes these methods computationally intractable.
To overcome these drawbacks, some newly improved approaches were presented recently by using convex optimization approaches,
such as {\citep{FuB95A}} and {\citep{ChenS98A}}.
Comparing with the conventional frequency domain method,
the convex optimization approaches are generally of a better numerical property
by virtue of the interior-point polynomial time algorithms{\citep{NN94}} and require no frequency sweep.
But unfortunately, none of these approaches is capable of solving large-scale SSV test
since they are all exponentially complex.
%the readily available convex optimization methods in literature,
%such as that proposed by Chen and Sugie in 1998, are exponentially complex and therefore
%unable to solve large-scale SSV analysis and synthesis problem.
Moreover, most of the widely used numerical algorithms,
such as the Projective Algorithm presented by {\citet{Projective}},
are also incapable to handle large-scale convex optimization problems.

Over the past few years, I have made several contributions in solving large-scale SSV analysis and synthesis
using convex optimization method.
By using S-procedure techniques, I have presented a new convex optimization-based
approach for the large-scale mixed SSV analysis and synthesis problem.
Based on the analysis of currently available methods when dealing with large-scale SSV analysis and synthesis problem,
I derived a sufficient condition of the feasibility of robust LMI (Linear Matrix Inequality)
by using S-procedure, and consequently, a less computational expensive SSV upper bound for
a large-scale dynamic system was presented.
The proposed approach {\citep{FYH03}} which is polynomial complex bases on state-space and requires no frequency sweeping.
Based on this result, I also gave an LMI criterion for the existence of a full-state feedback SSV controller.

\section{Current Research}
\subsection{Object-oriented parallel SDP algorithm and software for large-scale robust control analysis and synthesis}
Developed parallel-oriented algorithm for solving large-scale LMI (Linear Matrix Inequality)
and migrated serial code to parallel environment (MPI) using object-oriented technique(C++ and Fortran95)
with performance tuned for IBM RS/6000 SP3 architecture and Sun \& Linux cluster.

Over the past few years, convex optimization, and semidefinite programming (SDP) in particular, have
come to be recognized as valuable numerical tools for control system analysis and design.
A number of publications can be found in the control literature that survey applications of
SDP to the solution of system and control problems. In parallel, there has been considerable
recent research on algorithms and software for the numerical solution of SDPs.
%This interest was primarily motivated by applications of SDP in combinatorial
%optimization but, more recently, also by the applications in control.

\bibliographystyle{unsrt}
\bibliography{../../bib/bib_art,../../bib/bib_book,../../bib/bib_cnf,../../bib/bib_man}

\end{document}
